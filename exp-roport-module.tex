\documentclass{article}
\usepackage{amsmath}
\usepackage{amssymb}
\usepackage{ctex}
\usepackage[margin=2cm]{geometry} % 设置较窄的边距使文档宽一些
\usepackage{multirow} % 支持表格中的多行单元格
\usepackage{graphicx} % 用于插入图片
\usepackage{subcaption} % 支持子标题
\usepackage{float} % 支持 [H] 浮动体选项

\title{\heiti\zihao{2} 阴影云纹法测量等高线 }
\author{\songti  孙振川\quad PB23081463  王晨萱 \quad PB23331860 \\ 课程号  ME2009.01 }
\date{\today}

\begin{document}
    \maketitle

\begin{abstract}
    % 在这里填写实验摘要内容
\end{abstract}

\noindent{\textbf{关键词:} % 在此填写实验的关键词}

\section{实验目的}
\begin{enumerate}
    \item % 在此填写实验目的
\end{enumerate}

\section{实验器材}
\label{sec:equipment}
\begin{enumerate}
    \item % 在此填写实验器材
    \item % 在此填写实验器材
    \item % 在此填写实验器材
    \item % 在此填写实验器材
\end{enumerate}
\section{实验原理}
% 在此填写实验的基本原理

\section{实验步骤}
% 在此列出实验的详细步骤

\section{思考题}
% 在此填写思考题及其答案

\section{实验结果与分析}
% 在此填写实验数据和分析结果

\section{结论}
% 在此填写实验结论

% 其他部分可继续按需补充
\end{document}
