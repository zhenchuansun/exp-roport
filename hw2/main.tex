\documentclass[12pt, a4paper, oneside]{ctexart}
\usepackage{amsmath, amsthm, amssymb, bm, color, framed, graphicx, hyperref, mathrsfs}

\title{\textbf{实验力学作业 2}}
\author{孙振川 \quad PB23081463}
\date{\today}
\linespread{1.5}
\definecolor{shadecolor}{RGB}{241, 241, 255}
\newcounter{problemname}
\newenvironment{problem}{\begin{shaded}\stepcounter{problemname}\par\noindent\textbf{题目\arabic{problemname}. }}{\end{shaded}\par}
\newenvironment{solution}{\par\noindent\textbf{解答:}}{\par}
\newenvironment{note}{\par\noindent\textbf{题目\arabic{problemname}的注记。}}{\par}

\begin{document}

\maketitle

\begin{problem}
    从四种不同级别的压力表读出摩天大楼的顶层气压如下表,试给出最终测量结果(最佳值及其不确定度)
\begin{center}
  \begin{tabular}{|c|c|c|c|}
\hline 压力表 & 精度等级 & 量程 $(\mathrm{kPa})$ & 读数 $(\mathrm{kPa})$ \\
\hline A & 2.5 & 150 & 85 \\
\hline B & 2.5 & 300 & 87 \\
\hline C & 1.6 & 300 & 82 \\
\hline D & 1.0 & 500 & 83 \\
\hline
\end{tabular}  
\end{center}
\end{problem}

\begin{solution}
    设各压力表的读数分别为 $x_1, x_2, x_3, x_4$,则它们的标准不确定度分别为
    $$u(x_1)=2.5 \% \times 150=3.75 \mathrm{kPa}$$
    $$u(x_2)=2.5 \% \times 300=7.5 \mathrm{kPa}$$
    $$u(x_3)=1.6 \% \times 300=4.8 \mathrm{kPa}$$
    $$u(x_4)=1.0 \% \times 500=5 \mathrm{kPa}$$
    各压力表的权值分别为
    $$w_1=\frac{1}{u^2(x_1)}=\frac{1}{3.75^2}=0.0711 \mathrm{kPa}^{-2}$$
    $$w_2=\frac{1}{u^2(x_2)}=\frac{1}{7.5^2}=0.0178 \mathrm{kPa}^{-2}$$
    $$w_3=\frac{1}{u^2(x_3)}=\frac{1}{4.8^2}=0.0434 \mathrm{kPa}^{-2}$$
    $$w_4=\frac{1}{u^2(x_4)}=\frac{1}{5^2}=0.04 \mathrm{kPa}^{-2}$$
    以上权值归一化后为
    $$\tilde{w_1}=\frac{w_1}{\sum_{i=1}^4 w_i}=\frac{0.0711}{0.1723}=0.4129$$
    $$\tilde{w_2}=\frac{w_2}{\sum_{i=1}^4 w_i}=\frac{0.0178}{0.1723}=0.1033$$
    $$\tilde{w_3}=\frac{w_3}{\sum_{i=1}^4 w_i}=\frac{0.0434}{0.1723}=0.2519$$
    $$\tilde{w_4}=\frac{w_4}{\sum_{i=1}^4 w_i}=\frac{0.04}{0.1723}=0.2320$$
    所以最终测量结果为
    $$\bar{x}=\sum_{i=1}^4 \tilde{w_i} x_i=0.4129 \times 85+0.1033 \times 87+0.2519 \times 82+0.2320 \times 83=84.07 \mathrm{kPa}$$
    $$u(\bar{x})=\sqrt{\frac{1}{\sum_{i=1}^4 w_i}}=\sqrt{\frac{1}{0.1723}}=2.41 \mathrm{kPa}$$
    取 $k=1$ ,则最终测量结果为
    $$\bar{x} \pm \Delta x=84.07 \pm 2.41 \mathrm{kPa}$$
\end{solution}

\begin{problem}
对某一轴径进行等精密度测量 9 次,得到下表数据。请按规范程序处理测量结果,包括:(1)求最佳值和标准误差,(2)分别以 Malikov 和 Abbe-Helmert 准则判断有无系统误差(按数据列顺序),(3)以 Grubbs 准则(危险率 $1 \%$ )判断有无粗大误差,有则剔除,(4)写出测量结果最佳值及其不确定度(不确定度以极限误差表达,不必修正系差)。
\begin{center}
\begin{tabular}{|c|c|c|c|c|c|c|c|c|c|}
\hline 序号 & 1 & 2 & 3 & 4 & 5 & 6 & 7 & 8 & 9 \\
\hline $\mathrm{~L} / \mathrm{mm}$ & 24.774 & 24.778 & 24.771 & 24.780 & 24.772 & 24.777 & 24.773 & 24.775 & 24.774 \\
\hline
\end{tabular}
\end{center}
\end{problem}

\begin{solution}
    \subsection{最佳值和标准误差}
    设测量值为 $x_i(i=1,2, \ldots, 9)$ ,则最佳值为
    \begin{align*}
    \bar{x} &= \frac{1}{9} \sum_{i=1}^9 x_i \\
        &= \frac{24.774+24.778+24.771+24.780+24.772+24.777+24.773+24.775+24.774}{9}\\
        &=24.775~\mathrm{mm}
    \end{align*}
    标准误差为
    \begin{align*}
    s(\bar{x}) &= \sqrt{\frac{\sum_{i=1}^9\left(x_i-\bar{x}\right)^2}{9(9-1)}} \\
           &=0.0029~\mathrm{mm}
    \end{align*}  
    \subsection{Malikov 和 Abbe-Helmert 准则判断有无系统误差}
    Malikov 准则:$$
D=\sum_{i=1}^{\frac{n-1}{2}} d_i-\sum_{i=\frac{n+3}{2}}^n d_i=0.004~\mathrm{mm}<0.0065~\mathrm{mm}
$$
    Abbe-Helmert 准则:
    $$
\left|\sum_{i=1}^{n-1}\left(d_i \cdot d_{i+1}\right)\right|=6 \times 10^{-5} \mathrm{~mm}^2<\sqrt{n-1} \cdot \sigma^2=0.000076 \mathrm{~mm}^2
$$
    所以无系统误差。
    \subsection{Grubbs 准则判断有无粗大误差}
    计算 $G_1=\frac{\left|x_i-\bar{x}\right|_{max}}{s(\bar{x})}=\frac{0.005}{0.0029}=1.724<G_{\text {critical }}=2.32$ ,所以无粗大误差。
    \subsection{测量结果最佳值及其不确定度}
    取极限误差为$3s$ ,则测量结果为
    $$\bar{x} \pm \Delta x=24.775 \pm 0.0087 \mathrm{~mm}$$

\end{solution}

\begin{problem}
以下面关系式测定金属电导率 $\gamma=\frac{4 l}{\pi d^2 R}$ ,其中,$l$ 是导线长度,$R$ 是导线的电阻,$d$ 是导线直径,试问在怎样的测量条件下才能保证 $\gamma$ 有较小的测量误差?设 $\frac{\sigma_l}{l} 、 \frac{\sigma_d}{d} 、 \frac{\sigma_R}{R}$ 近似相等,分析 $l、d、R$ 中哪个值的误差对结果影响最大,从而测量得更准确?(提示:用 $\sigma_\gamma / \gamma$ 的表达式来分析)
\end{problem}
\begin{solution}

    由误差传递公式可知
    $$\frac{\sigma_\gamma}{\gamma}=\sqrt{\left(\frac{\partial \gamma}{\partial l} \frac{\sigma_l}{\gamma}\right)^2+\left(\frac{\partial \gamma}{\partial d} \frac{\sigma_d}{\gamma}\right)^2+\left(\frac{\partial \gamma}{\partial R} \frac{\sigma_R}{\gamma}\right)^2}$$
    计算各偏导数
    $$\frac{\partial \gamma}{\partial l}=\frac{4}{\pi d^2 R}$$
    $$\frac{\partial \gamma}{\partial d}=-\frac{8 l}{\pi d^3 R}$$
    $$\frac{\partial \gamma}{\partial R}=-\frac{4 l}{\pi d^2 R^2}$$
    所以
    $$\frac{\sigma_\gamma}{\gamma}=\sqrt{\left(\frac{\sigma_l}{l}\right)^2+\left(-2 \cdot \frac{\sigma_d}{d}\right)^2+\left(-1 \cdot \frac{\sigma_R}{R}\right)^2}$$
    设 $\frac{\sigma_l}{l} 、 \frac{\sigma_d}{d} 、 \frac{\sigma_R}{R}$ 近似相等为 $k$ ,则
    $$\frac{\sigma_\gamma}{\gamma}=\sqrt{1^2+(-2)^2+(-1)^2} k=\sqrt{6} k$$
    可见,$d$ 的误差对结果影响最大,所以应尽量减小 $d$ 的测量误差。

\end{solution}
\end{document}
