\documentclass[12pt, a4paper, oneside]{ctexart}
\usepackage[a4paper, left=2.5cm, right=2.5cm, top=2.5cm, bottom=2.5cm]{geometry}
\usepackage{amsmath, amsthm, amssymb, bm, color, framed, graphicx, hyperref, mathrsfs}
\usepackage{multirow}
\usepackage{graphicx,wrapfig}
\title{\textbf{实验力学作业 3}}
\author{孙振川 \quad PB23081463}
\date{\today}
\linespread{1.5}
\definecolor{shadecolor}{RGB}{241, 241, 255}
\newcounter{problemname}
\newenvironment{problem}{\begin{shaded}\stepcounter{problemname}\par\noindent\textbf{题目\arabic{problemname}. }}{\end{shaded}\par}
\newenvironment{solution}{\par\noindent\textbf{解答:}}{\par}
\newenvironment{note}{\par\noindent\textbf{题目\arabic{problemname}的注记。}}{\par}

\begin{document}

\maketitle
\begin{problem}
标定一个位移传感器,在 $0 \sim 5 \mathrm{~mm}$ 的测量范围内进行三个循环的测量,测量数据如下表,请给出该传感器的线性度(端点连线拟合直线)、灵敏度、最大滞环率和重复性误差。
\\
\begin{center}
\begin{tabular}{|l|l|l|l|l|l|l|l|}
\hline \multicolumn{2}{|l|}{输入(mm)} & 0.0 & 1.0 & 2.0 & 3.0 & 4.0 & 5.0 \\
\hline \multirow{6}{*}{输出(mV)} & 增序 & 0.0 & 4.3 & 9.2 & 14.4 & 20.6 & 25.0 \\
\hline & 减序 & 0.0 & 5.4 & 10.1 & 16.0 & 21.5 & 25.0 \\
\hline & 增序 & 0.0 & 5.0 & 9.8 & 15.1 & 19.8 & 25.0 \\
\hline & 减序 & 0.0 & 5.2 & 11.0 & 16.3 & 20.9 & 25.0 \\
\hline & 增序 & 0.0 & 5.1 & 10.2 & 15.3 & 20.5 & 25.0 \\
\hline & 减序 & 0.0 & 5.3 & 11.2 & 16.4 & 21.2 & 25.0 \\
\hline
\end{tabular}
\end{center}
\end{problem}

\begin{solution}
(1) 线性度计算如下:
\[\delta_L=\frac{\Delta_{\max }}{Y_{\mathrm{FS}}} \times 100 \%=\frac{1.5 \mathrm{~mV}}{25.0 \mathrm{~mV}} \times 100 \%=6 \%\]
(2) 灵敏度计算如下:
\[S=\frac{Y_{\mathrm{FS}}}{X_{\mathrm{FS}}}=\frac{25.0 \mathrm{~mV}}{5.0 \mathrm{~mm}}=5.0 \mathrm{~mV} / \mathrm{mm}\]
(3) 最大滞环率计算如下:
\[\delta_H=\frac{\Delta H_{\max }}{Y_{\mathrm{FS}}} \times 100 \%=\frac{1.6 \mathrm{~mV}}{25.0 \mathrm{~mV}} \times 100 \%=6.4 \%\]
(4) 重复性误差计算如下:
取各输入值对应的输出值的最大差值 $\Delta R_{\max }$ ,则有 $\Delta R_{\max }=\left\{10.1,11.0,11.2\right\} \text{的标准差}=0.586\mathrm{~mV}$ ,因此重复性误差为:
\[\delta_R=\frac{\Delta R}{Y_{\mathrm{FS}}} \times 100 \%=\frac{3 \time 0.586 \mathrm{~mV}}{25.0 \mathrm{~mV}} \times 100 \%=7.0 \%\]

\end{solution}
\newpage
\begin{problem}

\begin{wrapfigure}{r}{0.3\textwidth}
    \centering
    \includegraphics[width=0.17\textwidth]{q2.png}
    %\caption{多池串联}\label{多池串联}
\end{wrapfigure}
水银温度计如图:玻璃毛细管的内径 $\phi=0.2 \mathrm{~mm}$ ,内部封装 $V=0.1 \mathrm{ml}$ 的水银;水银的密度 $\rho=13.5 \mathrm{~g} / \mathrm{cm}^3$ 、比热容 $c=140 \mathrm{~J} /(\mathrm{kg} \cdot \mathrm{K})$ 、体积膨胀系数 $\alpha= 0.18 \% / K$ ;水银泡的传热表面积 $A=1.0 \mathrm{~cm}^2$ ,在充分接触下的表面传热系数 $h=20 \mathrm{~W} /\left(\mathrm{m}^2 \cdot \mathrm{~K}\right)$ 。问:

(1)该温度计的灵敏度是多少?(升高一度毛细管水银柱长度变化多少); 

(2)设温度计初始温度为 $T$ ,将其快速置入温度为 $T_a$的恒温水浴中,其动态响应的时间常数是多少?至少需要等待多少分钟读数才能保证动态误差小于初始温差的 $2 \%$ ?

注:表面传热系数有 $h=\frac{Q}{\Delta T \cdot A}=\frac{\mathrm{d} U / \mathrm{d} t}{\Delta T \cdot A}=\frac{\rho V c \cdot(\mathrm{~d} T / \mathrm{d} t)}{\left(T_a-T\right) \cdot A}$



\end{problem}


\begin{solution}
(1) 灵敏度计算如下:
\[S=\frac{\mathrm{d} l}{\mathrm{d} T}=\frac{\mathrm{d} V}{\mathrm{d} T} \cdot \frac{1}{A_c}=\alpha V \cdot \frac{4}{\pi \phi^2}=0.18 \% / K \cdot 0.1 \mathrm{ml} \cdot \frac{4}{\pi(0.2 \mathrm{~mm})^2}=5.73 \mathrm{~mm} / K\]
(2) 时间常数计算如下:
$$
\frac{\mathrm{d} T}{\mathrm{d} t} =\frac{h A}{\rho V c}\left(T_a-T\right)
$$
\[\tau=\frac{\rho V c}{h A}=\frac{13.5 \mathrm{~g} / \mathrm{cm}^3 \cdot 0.1 \mathrm{ml} \cdot 140 \mathrm{~J} /(\mathrm{kg} \cdot \mathrm{K})}{20 \mathrm{~W} /\left(\mathrm{m}^2 \cdot \mathrm{~K}\right) \cdot 1.0 \mathrm{~cm}^2}=0.945 \mathrm{~s}\]
动态误差小于初始温差的 $2 \%$ 所需时间计算如下:
\[t \geq -\tau \ln 0.02=4.56 \mathrm{~s}\]
\end{solution}
\newpage


\begin{problem}
某压电式压力传感器属于二阶系统,其固有频率为 $1 M H z$ ,阻尼比为 0.65。问:

(1)当测量阶跃压力时,该传感器的延迟时间、上升时间( $5 \% \sim 95 \%$ )与响应时间(与稳定值的相对误差小于 $3 \%$ )分别是多少?超调和衰减比分别是多少?

(2)当测量简谐压力波时,其谐振频率和最大振幅误差分别是多少?当谐振压力波频率为 0.5 MHz 时,其幅值误差和相位误差分别是多少?
\end{problem}
\begin{solution}
(1) 延迟时间、上升时间与响应时间(2\%)计算如下:
\[t_d=\frac{1+0.7\zeta}{\omega_n}=\frac{1+0.7\times 0.65}{1 M H z}=1.455 \mu s\]
\[t_r=\frac{\pi-\theta}{\omega_d}=\frac{\pi-\arccos 0.65}{0.76 M H z}=1.5 \mu s\]
\[t_s=\frac{4}{\zeta \omega_n}=\frac{4}{0.65 \times 1 M H z}=6.15 \mu s\]
超调和衰减比分别计算如下:
\[M_p=e^{-\frac{\zeta \pi}{\sqrt{1-\zeta^2}}}=e^{-\frac{0.65 \pi}{\sqrt{1-0.65^2}}}=0.163\]
\[S=\frac{1}{\sqrt{1-\zeta^2}}=\frac{1}{\sqrt{1-0.65^2}}=1.32\]
(2) 谐振频率和最大振幅误差计算如下:
\[\omega_r=\omega_n \sqrt{1-2 \zeta^2}=1 M H z \sqrt{1-2 \times 0.65^2}=0.66 M H z\]
\[M_r=\frac{1}{2 \zeta \sqrt{1-\zeta^2}}=\frac{1}{2 \times 0.65 \sqrt{1-0.65^2}}=1.02\]
当谐振压力波频率为 0.5 MHz 时,幅值误差和相位误差计算如下:
\[\frac{A_0}{A_i}=\frac{1}{\sqrt{\left(1-\left(\frac{\omega}{\omega_n}\right)^2\right)^2+\left(2 \zeta \frac{\omega}{\omega_n}\right)^2}}=\frac{1}{\sqrt{\left(1-\left(\frac{0.5 M H z}{1 M H z}\right)^2\right)^2+\left(2 \times 0.65 \times \frac{0.5 M H z}{1 M H z}\right)^2}}=1.09\]
\[\phi=-\arctan \frac{2 \zeta \frac{\omega}{\omega_n}}{1-\left(\frac{\omega}{\omega_n}\right)^2}=-\arctan \frac{2 \times 0.65 \times \frac{0.5 M H z}{1 M H z}}{1-\left(\frac{0.5 M H z}{1 M H z}\right)^2}=-40.4^{\circ}\]
\end{solution}

\end{document}