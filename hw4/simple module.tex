\documentclass[12pt, a4paper, oneside]{ctexart}
\usepackage{amsmath, amsthm, amssymb, bm, color, framed, graphicx, hyperref, mathrsfs}

\title{\textbf{实验力学作业 4}}
\author{孙振川 \quad PB23081463}
\date{\today}
\linespread{1.5}
\definecolor{shadecolor}{RGB}{241, 241, 255}
\newcounter{problemname}
\newenvironment{problem}{\begin{shaded}\stepcounter{problemname}\par\noindent\textbf{题目\arabic{problemname}. }}{\end{shaded}\par}
\newenvironment{solution}{\par\noindent\textbf{解答:}}{\par}
\newenvironment{note}{\par\noindent\textbf{题目\arabic{problemname}的注记。}}{\par}

\begin{document}

\maketitle
\begin{problem}
一批应变片的横向效应系数 $H=3 \%$ ,灵敏系数在材料泊松比 $\mu_0=0.2$ 的梁上标定。现将此应变片用于铝(泊松比 $\mu=0.36$ )试件应变测量,设有三个测点,应变片的安装方位和测点出应变状态分别使(a)$\varepsilon_l=\varepsilon_b$ ,(b)$\varepsilon_l=-\varepsilon_b$ , (c)$\varepsilon_b=-\mu \varepsilon_l$ ,试计算三种情况下由于横向效应造成的应变读数相对误差。
\end{problem}

\begin{solution}
    $$
\begin{aligned}
& K \cdot \varepsilon_{\text {read }}=K^* \cdot \varepsilon_l \\
& K=K_l \cdot\left(1-\mu_0 \boldsymbol{H}\right) \\
& K^*=K_l \cdot\left(1+\frac{\varepsilon_b}{\varepsilon_l} \cdot \boldsymbol{H}\right)
\end{aligned}
$$

 测量结果 $\varepsilon_i$ 满足:
 $$
 K \cdot \varepsilon_i=K^* \cdot \varepsilon_{x, i} \Rightarrow K_l \cdot\left(1-\mu_0 \cdot H\right) \cdot \varepsilon_i=K_l \cdot\left(1+\frac{\varepsilon_b}{\varepsilon_l}  \cdot H\right) \cdot \varepsilon_{x, i}
 $$
1. 当 $\varepsilon_l=\varepsilon_b$ 时,有
 $$
 \begin{aligned}
 & \left(1-\mu_0 \cdot H\right) \cdot \varepsilon_i=\left(1+H\right) \cdot \varepsilon_{x, i} \\
 & \Rightarrow \varepsilon_i=\frac{1+H}{1-\mu_0 \cdot H} \cdot \varepsilon_{x, i}=\frac{1+0.03}{1-0.2 \times 0.03} \cdot \varepsilon_{x, i} \approx 1.033 \cdot \varepsilon_{x, i}
 \end{aligned}
 $$
 相对误差约为 $3.3\%$ 。
 
 2. 当 $\varepsilon_l=-\varepsilon_b$ 时,有
 $$
 \begin{aligned}
 & \left(1-\mu_0 \cdot H\right) \cdot \varepsilon_i=\left(1-H\right) \cdot \varepsilon_{x, i} \\
 & \Rightarrow \varepsilon_i=\frac{1-H}{1-\mu_0 \cdot H} \cdot \varepsilon_{x, i}=\frac{1-0.03}{1-0.2 \times 0.03} \cdot \varepsilon_{x, i} \approx 0.967 \cdot \varepsilon_{x, i}
 \end{aligned}
 $$
 相对误差约为 $-3.3\%$ 。
 
 3. 当 $\varepsilon_b=-\mu \varepsilon_l$ 时,有
 $$
 \begin{aligned}
 & \left(1-\mu_0 \cdot H\right) \cdot \varepsilon_i=\left(1+\mu H\right) \cdot \varepsilon_{x, i} \\
 & \Rightarrow \varepsilon_i=\frac{1+\mu H}{1-\mu_0 \cdot H} \cdot \varepsilon_{x, i}=\frac{1+0.36 \times 0.03}{1-0.2 \times 0.03} \cdot \varepsilon_{x, i} \approx 1.010 \cdot \varepsilon_{x, i}
 \end{aligned}
 $$
    相对误差约为 $1.0\%$ 。
\end{solution}
\begin{problem}
采用阻值 $R=120 \Omega$ 、灵敏度系数 $K=2.0$ 的金属电阻应变片与阻值 $R=120 \Omega$的固定电阻组成电桥,供桥电压为 10 V。当应变片应变为 1000 时,若要使输出电压大于 10 mV,可采用何种工作方式?(设输出阻抗为无穷大)
\end{problem}

\begin{solution}
设应变片的应变为 $\varepsilon$ ,则应变片的阻值变化为
\[\Delta R=K R \varepsilon=2.0 \times 120 \Omega \times 1000 \times 10^{-6}=0.24 \Omega\]

如图 1,$R_1$ 、$R_3$ 为应变片,$R_2$ 、$R_4$ 为固定电阻。
\[U_O=\frac{U}{2} \frac{\Delta R}{R}=\frac{10 V}{2} \frac{0.24 \Omega}{120 \Omega}=10 mV=10 mV\]


\end{solution}
\begin{problem}
直流电桥供电电源电压 $U=3 V ; R_1$ 和 $R_2$ 为同型号的电阻应变片,初始电阻均为 $50 \Omega$ ,灵敏度系数 $K=2.0 ; R_3$ 和 $R_4$ 为固定电阻,$R_3=R_4=100 \Omega$ 。上述两应变片分别粘贴于等强度梁同一截面的正反两面,且和固定电阻组成卧式半等臂电桥。(1)请画出电桥电路图;(2)设等强度梁在受力后产生的应变为 5000,试求此时电桥输出端电压 $U_O$ 。
\end{problem}

\begin{solution}
(1) 电桥电路描述:
电桥电路图如下所示:
% \begin{center}
%     \includegraphics[width=0.5\textwidth]{circuit_diagram.png}
% \end{center}
(2) 设应变片 $R_1$ 和 $R_2$ 的应变均为 $\varepsilon=5000 \times 10^{-6}$ ,则应变片的阻值变化为
\[\Delta R=K R \varepsilon=2.0 \times 50 \Omega \times 5000 \times 10^{-6}=0.5 \Omega\]
因此,有
\[R_1=50 \Omega + 0.5 \Omega = 50.5 \Omega\]
\[R_2=50 \Omega - 0.5 \Omega = 49.5 \Omega\]
电桥输出端电压 $U_O$ 为
\[U_O=U \frac{R_4}{R_3+R_4}-U \frac{R_2}{R_1+R_2}=3 V \frac{100 \Omega}{100 \Omega + 100 \Omega}-3 V \frac{49.5 \Omega}{50.5 \Omega + 49.5 \Omega} = 15 mV\]
\begin{figure}[t]
    \centering
    \includegraphics[width=0.3\textwidth]{img1.png}
    \caption{电桥电路图}
\end{figure}


\end{solution}
\end{document}