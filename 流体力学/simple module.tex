\documentclass[12pt, a4paper, oneside]{ctexart}
\usepackage{amsmath, amsthm, amssymb, bm, color, framed, graphicx, hyperref, mathrsfs}

\title{\textbf{实验力学作业 2}}
\author{孙振川 \quad PB23081463}
\date{\today}
\linespread{1.5}
\definecolor{shadecolor}{RGB}{241, 241, 255}
\newcounter{problemname}
\newenvironment{problem}{\begin{shaded}\stepcounter{problemname}\par\noindent\textbf{题目\arabic{problemname}. }}{\end{shaded}\par}
\newenvironment{solution}{\par\noindent\textbf{解答:}}{\par}
\newenvironment{note}{\par\noindent\textbf{题目\arabic{problemname}的注记。}}{\par}

\begin{document}

\maketitle
\begin{problem}
采用控制体积微元法,推导柱坐标系下的微分形式动量方程
\end{problem}

\begin{solution}
设柱坐标系下的速度分量为$u_r, u_\theta, u_z$,密度为$\rho$,压力为$p$,体积力为$f_r, f_\theta, f_z$。考虑一个微小的柱坐标系控制体积,半径方向长度为$dr$,角度方向长度为$r d\theta$,轴向长度为$dz$。
1. 质量守恒方程:
\[\frac{\partial \rho}{\partial t} + \nabla \cdot (\rho \mathbf{u}) = 0\]
2. 动量守恒方程:
在$r$方向:
\[\rho \left( \frac{\partial u_r}{\partial t} + u_r \frac{\partial u_r}{\partial r} + \frac{u_\theta}{r} \frac{\partial u_r}{\partial \theta} + u_z \frac{\partial u_r}{\partial z} - \frac{u_\theta^2}{r} \right) = -\frac{\partial p}{\partial r} + \rho f_r\]
在$\theta$方向:
\[\rho \left( \frac{\partial u_\theta}{\partial t} + u_r \frac{\partial u_\theta}{\partial r} + \frac{u_\theta}{r} \frac{\partial u_\theta}{\partial \theta} + u_z \frac{\partial u_\theta}{\partial z} + \frac{u_r u_\theta}{r} \right) = -\frac{1}{r} \frac{\partial p}{\partial \theta} + \rho f_\theta\]
在$z$方向:
\[\rho \left( \frac{\partial u_z}{\partial t} + u_r \frac{\partial u_z}{\partial r} + \frac{u_\theta}{r} \frac{\partial u_z}{\partial \theta} + u_z \frac{\partial u_z}{\partial z} \right) = -\frac{\partial p}{\partial z} + \rho f_z\]
综上所述,柱坐标系下的微分形式动量方程为
\begin{align*}
&\rho \left( \frac{\partial u_r}{\partial t} + u_r \frac{\partial u_r}{\partial r} + \frac{u_\theta}{r} \frac{\partial u_r}{\partial \theta} + u_z \frac{\partial u_r}{\partial z} - \frac{u_\theta^2}{r} \right) = -\frac{\partial p}{\partial r} + \rho f_r, \\
&\rho \left( \frac{\partial u_\theta}{\partial t} + u_r \frac{\partial u_\theta}{\partial r} + \frac{u_\theta}{r} \frac{\partial u_\theta}{\partial \theta} + u_z \frac{\partial u_\theta}{\partial z} + \frac{u_r u_\theta}{r} \right) = -\frac{1}{r} \frac{\partial p}{\partial \theta} + \rho f_\theta, \\
&\rho \left( \frac{\partial u_z}{\partial t} + u_r \frac{\partial u_z}{\partial r} + \frac{u_\theta}{r} \frac{\partial u_z}{\partial \theta} + u_z \frac{\partial u_z}{\partial z} \right) = -\frac{\partial p}{\partial z} + \rho f_z.
\end{align*}

\end{solution}
\begin{problem}
采用控制体积微元法,推导柱坐标系下的微分形式能量方程
\end{problem}

\begin{solution}
设柱坐标系下的速度分量为$u_r, u_\theta, u_z$,密度为$\rho$,压力为$p$,内能为$e$,热传导率为$k$,体积力为$f_r, f_\theta, f_z$。考虑一个微小的柱坐标系控制体积,半径方向长度为$dr$,角度方向长度为$r d\theta$,轴向长度为$dz$。
能量守恒方程:
\[\rho \left( \frac{\partial e}{\partial t} + u_r \frac{\partial e}{\partial r} + \frac{u_\theta}{r} \frac{\partial e}{\partial \theta} + u_z \frac{\partial e}{\partial z} \right) = -p \left( \nabla \cdot \mathbf{u} \right) + \nabla \cdot (k \nabla T) + \rho (f_r u_r + f_\theta u_\theta + f_z u_z)\]
其中,$\nabla \cdot \mathbf{u}$在柱坐标系下表示为:
\[\nabla \cdot \mathbf{u} = \frac{1}{r} \frac{\partial (r u_r)}{\partial r} + \frac{1}{r} \frac{\partial u_\theta}{\partial \theta} + \frac{\partial u_z}{\partial z}\]
综上所述,柱坐标系下的微分形式能量方程为
\[\rho \left( \frac{\partial e}{\partial t} + u_r \frac{\partial e}{\partial r} + \frac{u_\theta}{r} \frac{\partial e}{\partial \theta} + u_z \frac{\partial e}{\partial z} \right) = -p \left( \frac{1}{r} \frac{\partial (r u_r)}{\partial r} + \frac{1}{r} \frac{\partial u_\theta}{\partial \theta} + \frac{\partial u_z}{\partial z} \right) + \nabla \cdot (k \nabla T) + \rho (f_r u_r + f_\theta u_\theta + f_z u_z)\]


\end{solution}


\title{柱坐标系下能量方程的推导}
\author{}
\date{}
\maketitle

\section*{推导过程}

首先,考虑流体在柱坐标系下的控制体积微元。柱坐标系的坐标为 \((r, \theta, z)\),其中 \(r\) 是径向坐标,\(\theta\) 是角坐标,\(z\) 是高度坐标。

对于控制体积微元,能量的变化率包含了三项:

1. **能量的传递**:通过对流(流体的流动)和传导(热的传递)。
2. **能量的输入和输出**:包括外力做功和热源。
3. **局部的能量积累**:流体中能量的增加或减少。

能量方程的微分形式可表示为:

\[
\frac{\partial}{\partial t} \left( \rho E \right) + \nabla \cdot \left( \rho E \mathbf{v} \right) = \nabla \cdot \left( k \nabla T \right) + \Phi + Q
\]

其中:
- \(\rho E\) 是单位体积的总能量,\(\rho\) 是流体密度,\(E = e + \frac{v^2}{2}\) 是总能量(包括内部能和动能),其中 \(e\) 是内部能,\(v\) 是流速。
- \(\mathbf{v}\) 是流速矢量。
- \(k\) 是热导率。
- \(\Phi\) 是粘性耗散项(由于粘性引起的能量损失)。
- \(Q\) 是外部热源。

接下来,我们在柱坐标系下展开各个项。

\subsection*{对流项}

对流项 \(\nabla \cdot \left( \rho E \mathbf{v} \right)\) 展开为:

\[
\nabla \cdot \left( \rho E \mathbf{v} \right) = \frac{1}{r} \frac{\partial}{\partial r} \left( r \rho E v_r \right) + \frac{1}{r} \frac{\partial}{\partial \theta} \left( \rho E v_\theta \right) + \frac{\partial}{\partial z} \left( \rho E v_z \right)
\]

其中,\(v_r\), \(v_\theta\), \(v_z\) 分别是径向、角向和垂直方向上的速度分量。

\subsection*{热传导项}

热传导项 \(\nabla \cdot \left( k \nabla T \right)\) 展开为:

\[
\nabla \cdot \left( k \nabla T \right) = \frac{1}{r} \frac{\partial}{\partial r} \left( r k \frac{\partial T}{\partial r} \right) + \frac{1}{r} \frac{\partial}{\partial \theta} \left( k \frac{\partial T}{\partial \theta} \right) + \frac{\partial}{\partial z} \left( k \frac{\partial T}{\partial z} \right)
\]

\subsection*{粘性耗散项}

粘性耗散项 \(\Phi\) 可通过以下公式表示:

\[
\Phi = \mu \left( \frac{\partial v_r}{\partial r} + \frac{1}{r} \frac{\partial v_\theta}{\partial \theta} + \frac{\partial v_z}{\partial z} \right)^2
\]

其中,\(\mu\) 是流体的粘度。

\subsection*{外部热源项}

外部热源项 \(Q\) 是体积单位的热源。

\subsection*{最终的能量方程}

将上述项代入能量方程中,得到柱坐标系下的能量方程的微分形式:

$$
\frac{\partial}{\partial t} \left( \rho E \right) + \frac{1}{r} \frac{\partial}{\partial r} \left( r \rho E v_r \right) + \frac{1}{r} \frac{\partial}{\partial \theta} \left( \rho E v_\theta \right) +   $$$$ 
\frac{\partial}{\partial z} \left( \rho E v_z \right) = \frac{1}{r} \frac{\partial}{\partial r} \left( r k \frac{\partial T}{\partial r} \right)   \\  $$$$
+ \frac{1}{r} \frac{\partial}{\partial \theta} \left( k \frac{\partial T}{\partial \theta} \right) + \frac{\partial}{\partial z} \left( k \frac{\partial T}{\partial z} \right) + \Phi + Q $$ 
\end{document}