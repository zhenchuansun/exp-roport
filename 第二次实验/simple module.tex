\documentclass{article}
\usepackage{amsmath}
\usepackage{amssymb}
\usepackage{ctex}
\usepackage[margin=2cm]{geometry} % 设置较窄的边距使文档宽一些
\usepackage{multirow} % 支持表格中的多行单元格
\usepackage{graphicx} % 用于插入图片
\usepackage{subcaption} % 支持子标题
\usepackage{float} % 支持 [H] 浮动体选项
\title{\heiti\zihao{2} 动态应变、动态加速度测量  }
\author{\songti  孙振川\quad PB23081463  王晨萱 \quad PB23331860 \\
课程号  ME2009.01 }
\date{\today}
\begin{document}
    \maketitle
\begin{abstract}
   
    通过本次实验,学习了动态应变测量的基本方法,掌握了 DH5922D 动态信号测试分析系统的使用,并熟练掌握了用 DH5922D 动态信号测试分析系统采集动态应变信号的方法。实验中通过扫频信号源激振简支梁,利用应变片测量了梁在不同位置的动态应变信号,并记录了各测点的周期、频率及峰峰值。实验结果显示,各通道应变波形的相位相同,但最大值不同,反映了不同位置的受力情况。通过本次实验,加深了对动态应变测量原理和方法的理解,提高了实际操作能力。
    
    \noindent{\textbf{关键词: } 动态应变测量;DH5922D 动态信号测试分析系统}
\end{abstract}
\section{实验目的}
\begin{enumerate}
    \item 学习动态应变测量的基本方法;
    \item 掌握 DH5922D 动态信号测试分析系统的使用;
    \item 熟练掌握用 DH5922D 动态信号测试分析系统采集动态应变信号。
    
\end{enumerate}



\section{实验器材}
\label{sec:equipment}
    \begin{enumerate}
        \item DH1301 扫频信号源 
        \item DH5922D 四通道动态信号测试分析系统 
        \item 简支梁 
        \item 激振器
        \item IEPE 压电式加速度传感器
        \item 电脑及采集软件 
    \end{enumerate} 

\section{实验原理}
\subsection{ 实验装置 }

\begin{figure}[H]
    \centering
    \includegraphics[width=0.8\textwidth]{img1.png}
    \caption{测量动态应变和加速度的实验装置示意图 }
    \label{fig:star_connection_circuit}
\end{figure}
\subsection{ 应变实验原理 }
扫频信号源输出频率可连续变化的正弦信号至激振器,激振器产生正弦激振力
给简支梁加载,梁发生振动即产生动态的应变信号。贴于梁上的应变片与应变信号
输入线组成应变电桥,将应变信号输入线接入数据采集仪的接口,启动采集软件,
采集动态应变信号。
\subsection{ 加速度实验原理 }
扫频信号源输出频率可连续变化的正弦信号至激振器,激振器产生正弦激振力
给简支梁加载。梁发生振动即产生动态的加速度。IEPE 压电式加速度传感器受振动
产生加速度信号(IEPE 加速度传感器输出的是电压量),数据采集系统采集记录动
态加速度信号,供精确测量。
\subsection{ 加速度实验标定  }
数采系统采集的加速度是电压值(单位:V),要得到被测量的物理量(加速度
单位:m/s²),必须进行标定。可根据加 IEPE 加速度传感器的编号,在产品合格证上
找到其轴向灵敏度,在通道参数设置时填入其灵敏度即可。 
\section{实验内容}
\subsection{接桥:}
在简支梁上分别选取 3 个工作片,接电桥,桥路可选择 1/4 桥接法
\begin{figure}[H]
    \centering
    \includegraphics[width=0.8\textwidth]{img3.png}
    \caption{电路接桥示意图 }
\end{figure}

\subsection{动态应变测量:}
\begin{enumerate}
    \item 将应变信号输入线接数据采集仪的通道接口
    \item 打开 DH5922D 数据采集仪的电源开关,等采集仪的等待指示灯熄灭后,打开数
据采集软件
    \item 在采集软件中依次进行新建文件夹、通道参数设置、存储规则设置,测量设置(注
意:通道一定要进行平衡清零,选择合适的采样率)
    \item 扫频信号源的功率输出接激振器,打开信号源的电源开关,将扫频信号源的
输出电压调至 1.5V 左右,扫频信号源的的频率调置 130HZ 左右,对简支梁进行激
振。
    \item 微调扫频信号源的频率,使屏幕上出现稳定、幅度适中的动态应变波形,采
集动态应变信号并记录动态应变信号的周期、频率及峰峰值,将动态应变信号截图
保存
\end{enumerate}

\section{测量结果}

\begin{table}[H]
\centering
\caption{各测点应变数据}
\begin{tabular}{|c|c|c|c|c|}
\hline 通道 & 应变片位置 & 周期(ms) & 频率(Hz) & 应变峰峰值( $\mu \varepsilon$ ) \\
\hline CH2 &偏上 1/4 处 &9.54 & 104.82&7.925 \\
\hline CH3 &正上方 &9.52 & 105.04& 15.989\\
\hline CH4 & 正下方&9.50 & 105.26& 15.671\\
\hline
\end{tabular}
\end{table}


\section{预习思考题 }
\subsection{分析各通道的应变波形,各通道应变波形的相位、最大值之间的相互关系如何?为什么?  }
各通道应变波形的相位相同,最大值不同。因为各应变片位置不同,受力情况不同,导致测得的应变峰峰值不同。


\subsection{ 画图表示梁横截面上弯曲正应力的分布规律。}
\begin{figure}[H]
    \centering
    \includegraphics[width=0.5\textwidth]{img4.png}
    \caption{梁横截面上弯曲正应力的分布规律示意图 }
\end{figure}


\subsection{ 加速度信号的频率与激振信号的频率有什么关系?}
加速度信号的频率与激振信号的频率相同。


\end{document}