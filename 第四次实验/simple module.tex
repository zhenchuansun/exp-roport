\documentclass{article}
\usepackage{amsmath}
\usepackage{amssymb}
\usepackage{ctex}
\usepackage[margin=2cm]{geometry} % 设置较窄的边距使文档宽一些
\usepackage{multirow} % 支持表格中的多行单元格
\usepackage{graphicx} % 用于插入图片
\usepackage{subcaption} % 支持子标题
\usepackage{float} % 支持 [H] 浮动体选项
\title{\heiti\zihao{2} 阴影云纹法测量等高线 }
\author{\songti  孙振川\quad PB23081463  王晨萱 \quad PB23331860  高伟佳 \quad PB23051048 \\
课程号  ME2009.01 }
\date{\today}
\begin{document}
    \maketitle
\begin{abstract}
    本实验采用阴影云纹法测量物体表面等高线,通过平行光经基准栅投射至试件形成干涉条纹,利用条纹级数与高度的关系式 $h_N = NLp / D$ 计算各点高度。实验器材包括光源、基准栅、试件和数码相机,步骤为布置共面光路、调整至条纹清晰、拍摄多组条纹图并计算高度。实验结果显示不同距离下的条纹图反映表面形貌,测点高度平均值为 $9.904\,\text{mm}$。误差主要源于光路几何配置、光源平行性及光栅稳定性,可通过校准光路、使用平行单色光、固定光栅及优化算法减小。该方法适用于全场非接触测量,可用于离面变形分析(条纹移动量 $\Delta z \approx p/(2\theta)$)。实验加深了对阴影云纹法原理与技术的理解。


  \noindent{\textbf{关键词:} 阴影云纹法;等高线;条纹级数;高度测量}
\end{abstract}
\section{实验目的}
\begin{enumerate}
    \item 掌握阴影云纹法测量等高线的原理和技术
\end{enumerate}



\section{实验器材}
\label{sec:equipment}
    \begin{enumerate}
        \item 光源
        \item 基准栅
        \item 试件 
        \item 数码相机
    \end{enumerate} 

\section{实验原理}
\begin{figure}[H]
    \centering
    \includegraphics[width=0.5\textwidth]{img1.png}
    \caption{实验原理图点照射、点接受阴影莫尔条纹原理 }
\end{figure}

光源 S 照射基准栅板 M,将栅线投影到被测物体 O 上,形成试件栅。由于物体表面不平,使栅线的影子畸变。当观察点在 C 处时,由于两栅的几何干涉效应可看到物体表面出现明暗相间的条纹。如果 CS 平行于基准栅 M,D、L 远大于物体尺寸,则这些条纹是一些等高线。其距离基准栅的高度由下式确定:

$$
\mathrm{h}_{\mathrm{N}}=\mathrm{NLp} / \mathrm{D}
$$


因此,只要测量出 $\mathrm{L} 、 \mathrm{p} 、 \mathrm{D}$ ,就可计算出各级数条纹的高度。

\section{实验步骤}

\begin{enumerate}
    \item 按照原理图布置光路,注意使物体、光源和相机在同一水平面上,基准栅与
试件最高点相接触。
    \item 观察等高线图。打开光源,调整光源及相机的角度,使试件表面出现十条左右
条纹,并且阴影条纹清晰,不受光源反射象的影响。
    \item 拍摄等高线图。改变光路参数,使试件上特征点处于亮(或暗)条纹的中心,
但级数不同。拍摄几次。 
    \item 计算等高线的高度。将数码相机的图象传至微机,打印。在云纹图上确定特征点的条纹级数,注意整数级与半级条纹的判别。计算特征点的高度$h_N$
\end{enumerate}

\section{测量结果}
\begin{figure}[H]
    \centering
    \includegraphics[width=0.5\textwidth]{img1.jpg}
    \caption{D=51.32mm 时的条纹图 }
\end{figure}
\section{测量结果}
\begin{figure}[H]
    \centering
    \includegraphics[width=0.5\textwidth]{img2.jpg}
    \caption{D=49.11mm 时的条纹图 }
\end{figure}
\section{测量结果}
\begin{figure}[H]
    \centering
    \includegraphics[width=0.5\textwidth]{img3.jpg}
    \caption{D=46.34mm 时的条纹图 }
\end{figure}


\begin{table}[H]
\centering
\caption{各测点应变数据}
\begin{tabular}{|l|l|l|l|l|l|}
\hline 节距 P(mm) & L(mm) & D(mm) & N & $\mathrm{h}_{\mathrm{n}}(\mathrm{mm})$ & $\mathrm{h}_{\mathrm{n}}(\mathrm{mm})$平均值 \\
\hline \multirow{3}{*}{0.5} & 74.56& 51.32& 14& 10.189& \multirow{3}{*}{9.904} \\
\hline &74.56 & 49.11& 13& 9.868& \\
\hline &74.56 & 46.34& 12& 9.654& \\
\hline
\end{tabular}
\end{table}


\section{预习思考题 }
\subsection{用阴影云纹法测量高度的最大误差来源是什么?应如何避免? }
阴影云纹法测量高度时,\textbf{最大的误差来源是参考光栅与被测表面之间的几何配置误差,尤其是两者不平行所导致的系统误差}。此外,光源的平行性和单色性不足、光栅质量与安装不稳定、以及相位解调算法误差也会影响测量精度。

为减小误差,可采取以下措施:
\begin{itemize}
  \item \textbf{保证光路几何精度}:调整参考光栅与被测表面尽量平行,精确校准光源、光栅与相机的相对位置;
  \item \textbf{使用平行光与单色光源}:采用准直光源(如激光)以减小入射角差异,使用单色光提高条纹对比度;
  \item \textbf{选用高质量光栅并固定牢固}:避免光栅变形、振动或位移;
  \item \textbf{优化相位解调算法}:采用多步相移法或傅里叶变换法,并通过标定提高测量精度。
\end{itemize}
\subsection{如何利用阴影云纹法测量离面变形? }
阴影云纹法可通过记录被测物体在受力或温度等外界因素作用下,\textbf{表面因离面位移(out-of-plane deformation)引起的阴影光栅图案变化},进而获取离面位移场或变形信息。当物体表面发生垂直于表面的位移(即 $z$ 方向位移)时,阴影与参考光栅之间产生的干涉条纹会发生移动,该条纹移动量与离面位移存在定量关系。

\paragraph{测量方法:}
\begin{enumerate}
  \item 布置光路:在物体前方或后方放置已知周期的参考光栅,用平行光从一定角度照射,形成阴影与光栅叠加的云纹图案;
  \item 记录初始云纹图:未施加变形前,拍摄初始状态的云纹图像;
  \item 施加变形并记录:对物体施加载荷(如力、热),引起离面变形后,再次拍摄云纹图;
  \item 分析条纹移动:通过对比变形前后云纹图中条纹的位置变化,计算各点的离面位移。
\end{enumerate}

\paragraph{定量关系:}
当物体发生离面位移 $\Delta z$ 时,云纹条纹会相应移动。其关系为:
\[
\Delta z \approx \frac{p}{2 \theta}
\]
其中:
\begin{itemize}
  \item $\Delta z$:离面位移(沿光轴,通常为 $z$ 方向);
  \item $p$:光栅节距;
  \item $\theta$:光源入射光与光栅法线的夹角。
\end{itemize}
条纹移动一个周期(即光栅节距 $p$)对应离面位移 $\Delta z = \frac{p}{2 \tan \theta} \approx \frac{p}{2 \theta}$(小角度近似)。

\paragraph{优点与适用:}
阴影云纹法适用于\textbf{大变形、全场、非接触测量},对表面状态要求低,常用于\textbf{结构力学实验、热变形、振动分析}等领域。

\end{document}